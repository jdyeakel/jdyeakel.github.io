\documentclass[margin,line,12pt]{res}

% Using EB Garamond font
\usepackage{ebgaramond}


%\usepackage{etaremune}
\usepackage{hyperref}
\usepackage[usenames, dvipsnames]{color}

\oddsidemargin -.5in
\evensidemargin -.5in
\textwidth=6.0in
\itemsep=0in
\parsep=0in
% if using pdflatex:
%\setlength{\pdfpagewidth}{\paperwidth}
%\setlength{\pdfpageheight}{\paperheight}

\newenvironment{list1}{
  \begin{list}{\ding{113}}{%
      \setlength{\itemsep}{0in}
      \setlength{\parsep}{0in} \setlength{\parskip}{0in}
      \setlength{\topsep}{0in} \setlength{\partopsep}{0in}
      \setlength{\leftmargin}{0.17in}}}{\end{list}}
\newenvironment{list2}{
  \begin{list}{$\bullet$}{%
      \setlength{\itemsep}{0in}
      \setlength{\parsep}{0in} \setlength{\parskip}{0in}
      \setlength{\topsep}{0in} \setlength{\partopsep}{0in}
      \setlength{\leftmargin}{0.2in}}}{\end{list}}

\newcommand\reverselabel[1]{%
  \def\theenumi{}%
  \renewcommand\makelabel{\makebox[\dimexpr\labelwidth-3pt\relax][r]{%
    \the\numexpr#1-\value{enumi}+1\relax}}}%


\begin{document}

\name{Justin D. Yeakel, Ph.D. \vspace*{.2in}}

\begin{resume}

\section{\sc Current Positions}
{\bf University of California, Merced}, Merced, CA USA\\
Biology Program Chair \hfill {\bf July, 2023 - present} \\
Associate Professor \hfill {\bf July, 2021 - present} \\
Assistant Professor \hfill {\bf January, 2016 - 2021} \\
Assistant Research Scientist \hfill {\bf 2015 - 2016}\\ \\
{\bf Santa Fe Institute}, Santa Fe,  NM USA\\
External Professor \hfill {\bf August, 2024 - present}

\section{\sc Contact Information}
% \vspace{.05in}
% \begin{tabular}{@{}p{2in}p{4in}}
Science \& Engineering 1, 288             \hfill {\it Voice:}  (209) 285-9571 \\
Life \& Environmental Sciences   \hfill {\it E-mail:}  jyeakel@ucmerced.edu \\
University of California, Merced  \hfill {\it Web:} http://jdyeakel.github.io\\
Merced, CA 95340, USA  \hfill  \\
% \end{tabular}


\section{\sc Research Interests}
Paleoecology,
Food webs,
Stable isotopes,
Foraging dynamics,
Niche construction,
Community assembly,
Ecosystem engineering,
Human evolution

\section{\sc Past Positions}
{\bf Santa Fe Institute}, Santa Fe, NM USA

\vspace{-.3cm}
{\em Omidyar Fellow} \hfill {\bf June, 2014 - 2017}\\
%Includes current Ph.D.~research, Ph.D.~and Masters level coursework and research/consulting projects.
\vspace{-.3cm}

{\bf Simon Fraser University}, Vancouver, BC Canada

\vspace{-.3cm}
{\em Postdoctoral Researcher} \hfill {\bf June, 2012 - 2014}\\


\section{\sc Education}
{\bf University of California, Santa Cruz}, Santa Cruz, CA USA\\
%{\em Department of Statistics}
\vspace*{-.1in}
\begin{list1}
\item[] Ph.D. Ecology \& Evolutionary Biology \hfill {\bf 2006 - 2012}\\
\begin{list2}
\vspace*{-.08in}
\item Dissertation Topic: ``The structure of mammalian food webs: Interpreting, predicting, and updating estimates of species interactions in paleontological and modern communities''
%\item Dissertation Topic:  ``Hierarchical Models for Multiple Ratings
%  in Performance-Based\\ \hspace*{1.23in} Student Assessments.''
\item Advisor troika: Paul L. Koch (Earth \& Planetary Sciences), Marc Mangel (Applied Mathematics), James A. Estes (Ecology \& Evolutionary Biology).\\ External committee member: Paulo R. Guimar\~aes Jr. (University of Sao P\~aulo)
\end{list2}
% \vspace*{.05in}
% \item[] M.S., Statistics,  May 2000
\end{list1}

% {\bf Duke University}, Durham, North Carolina USA\\
% %{\em Department of Mathematics and Statistics}
% \vspace*{-.1in}
% \begin{list1}
% \item[] M.S., Botany (Ecology),  May, 1998
% \end{list1}

{\bf Kent State University}, Kent, OH USA\\
%{\em Department of Mathematics and Statistics}
\vspace*{-.1in}
\begin{list1}
\item[] B.S. Biological Anthropology (Biology minor),  May, 2004 \hfill {\bf 1999 - 2004} \\
\emph{Summa cum laude}
\end{list1}

\section{\sc Awards \& Honors}
\begin{itemize}
  \item \emph{Developing or Improving Academic Programs \& Tracks} \hfill Fall 2024 \\ UC Merced School of Natural Sciences
\end{itemize}

% %\vspace{-.1cm}
% {\em Instructor} \hfill {\bf May - June, 2002}\\
% Co-taught graduate level course for the Master of Science in
% Computational Finance program.  Shared responsibility for lectures, exams,
% homework assignments, and  grades.
% \vspace*{.05in}
% \begin{list2}
% \item 46-731 Probability and Statistics, Summer 2002.
% \end{list2}
%
%
% %\vspace{-.1cm}
% {\em NSF VIGRE Teaching Fellow} \hfill {\bf January - May, 2001}\\
% Head teaching assistant.
% Duties included  shared administrative responsibilities with faculty
% instructor, fielding of all student inquiries, and oversight of
% graduate student teaching assistants and graders.
% \vspace*{.05in}
% \begin{list2}
% \item 36-217 Probability Theory and Random Processes, Spring 2001.
% \end{list2}
%
% %\vspace{-.1cm}
% {\em Teaching Assistant} \hfill {\bf August, 2001  - present}\\
% Duties at various times have included
% office hours and leading weekly computer lab exercises.


% \section{\sc Fellowships \& Grants -- \\  In Revision}
% \begin{itemize}
% \item National Science Foundation, Biological Oceanography: \emph{Collaborative Research: Resilience and collapse in marine food webs across paleo, historical, and modern ecological time scales.}\\
% Role: PI\\
% Amount: TBD\\
% Status: \emph{In revision}
% \end{itemize}

% \vspace{5mm}

% \section{\sc Fellowships \& Grants -- \\  In Review}
% \begin{itemize}


% % \item National Science Foundation, Organismal Response to Climate Change: \emph{Collaborative Research: ORCC: Evaluating the impact of precipitation variability on foraging and fitness in desert rodents.}\\
% % Role: PI


% % \item National Science Foundation, Integrative Organismal Systems: \emph{Collaborative Research: Linking gut microbiota to host nutrient dynamics, immunity, and survival in a resource-limited ecosystem.} Role: PI


% % \item W. M. Keck Foundation: \emph{The ecological dynamics of mammalian macroevolution across the Cenozoic.} Role: PI


% \item National Science Foundation, NRT-HDR: \emph{Data-integrated physical modeling for sustainability applications.} Role: Collaborator



% \end{itemize}


% \begin{itemize}
% \item Hellman's Fellowship. (2020)\\ 
% Role: Single-PI.\\
% Amount: \$50000.\\
% Status: \emph{In Review}

% \item Blavatnik Young Investigator's Fellowship. (2020)\\
% Role: Single-PI.\\ 
% Amount: \$200000.\\
% Status: \emph{In Review}

% \item National Science Foundation Postdoctoral Fellowship to Brian Tanis (2020)\\
% Role: co-Mentor\\
% Amount: \$252000\\
% Status: \emph{In Review}

% \item National Science Foundation, BIO-Division of Environmental Biology: \emph{Collaborative Research: Mapping foraging to fitness: quantifying the causes and consequences of resource selection in a desert rodent community.}\\
% Role: PI\\
% Amount: \$1,407,795\\ %\$289525\\
% Status: \emph{In review}


% \item National Science Foundation, OCE-Biological Oceanography: \emph{Collaborative Research: Resilience and collapse in marine food webs across paleo, historical, and modern ecological time scales.}\\
% Role: PI\\
% Amount: \$1,619,786\\ %\$668110\\
% Status: \emph{In review}

% \end{itemize}


\section{\sc Fellowships \& Grants -- \\ Awarded}
\begin{itemize}
\item National Science Foundation, Emerging Mathematics in Biology: \emph{Collaborative Research: Reconstructing Interactions of Species Ensembles (RISE): Generalized dynamical frameworks for exploring the structure and function of ancient marine communities through time.} Role: Lead PI


\item National Science Foundation, LTER Renewel: \emph{LTER: Sevilleta Site: Environmental variability at dryland ecotones.} Role: Collaborator

\item UC Merced COR: \emph{Unraveling ecological drivers of mammalian macroevolution.} (2024) Role: PI w/ Anna Carolina de Almeida; Amount: \$10,000; Status: \emph{Awarded}

\item UC Merced COR: \emph{Exploring the relationships between morphological variation, dietary breadth, and patterns of extinction among Hawaiian Honeycreepers (Drepanididae).} (2022) Role: PI w/ Irina Birskis Barros; Amount: \$5,000; Status: \emph{Awarded}
  
\item National Science Foundation, LTREB: \emph{Collaborative Research: Experimental determination of trophic dynamics and energy flows in a semiarid habitat in Chile.} (2020) Role: PI; Lead PI: Doug Kelt (UC Davis); Amount: \$356,376; Status: \emph{Awarded}

% \item National Science Foundation, NRT: \emph{Unraveling the response of mammalian communities to grassland expansion: A neural network approach to resolving past and present food webs} 2018.  \$323,260. \emph{Status: In Review.}
\item Sevilleta Summer Fellowship. (2019) Role: Single-PI; Amount: \$5,000; Status: \emph{Awarded}

\item National Science Foundation, LTER: \emph{Sevilleta (SEV) Site: Climate Variability at Dryland Ecotones.} (2018) Role: Senior Collaborator; Lead PI: Jennifer Rudgers (U New Mexico); Amount: \$6,400,000; Status: \emph{Awarded}

\item National Science Foundation, SGP-Sedimentary Geology \& Paleobiology: \emph{Assessing millennial-scale community stability using highly-resolved mammal and vegetation food webs.} (2016) Role: co-PI; Amount: \$1,144,448; Status: \emph{Awarded}

\item James S. McDonnell Complex Systems Postdoctoral Fellowship to J.P. Gibert (2016-2018) Role: Mentor; Amount: \$200,000; Status: \emph{Awarded}

%COUPLEDGRASSLANDS WORKSHOP GRANT
%15K
\item Santa Fe Institute Working Group grant: \emph{Coupled Grassland and Mammalian Community Dynamics over Ecological and Evolutionary Timescales} (2015) Role: Principle Organizer; Amount: \$15,000; Status: \emph{Awarded}

%LIFEINVESTMENT STRATEGIES GRANT
\item Santa Fe Institute Workshop grant: \emph{The Evolutionary Ecology of Complex Life Investment Strategies} (2015) Role: Principle co-Organizer; Amount: \$25,000; Status: \emph{Awarded}
%25K

\item Omidyar Postdoctoral Fellowship, Santa Fe Institute. (2014); Amount: \$210,000; Status: \emph{Awarded}

\item UC Santa Cruz Regents Fellowship. (2011) Amount: \$46000; Status: \emph{Awarded}

\item UC Santa Cruz Deans Fellowship. (2010) Amount: \$46000; Status: \emph{Awarded}

\item National Science Foundation Graduate Research Fellowship. (2006) Amount: \$138000; Status: \emph{Awarded}

% \item UC Santa Cruz Regents Fellowship. (2011) \\ 
% Amount: \$46000\\
% Status: \emph{Awarded}

% \item UC Santa Cruz Dissertation-Year Fellowship. (2011)\\
% Status: \emph{Runner-up}

% \item UC Santa Cruz Deans Fellowship. (2010)\\
% Amount: \$46000\\
% Status: \emph{Awarded}

% \item Institute of Geophysics and Planetary Physics (IGPP) Grant. (2008)\\
% Amount: \$5000\\
% Status: \emph{Awarded}

% \item Society of Vertebrate Paleontology Travel Award. (2007)\\
% Amount: \$300\\
% Status: \emph{Awarded}

% \item UC Santa Cruz Graduate Research Symposium. (2007)\\
% Status: \emph{Honorable Mention}

% \item Committee on Research Grant- SRG. (2007)\\
% Amount: \$11960
% Status: \emph{Awarded}

% \item Committee on Research Grant- FRG. (2007)\\
% Amount: \$2500\\
% Status: \emph{Awarded}

% \item Friends of Long Marine Lab Research Grant. (2006)\\
% Amount: \$800\\
% Status: \emph{Awarded}

% \item Department of Anthropology Internal Research Grant. (2006)\\ 
% Amount: \$800\\
% Status: \emph{Awarded}

% \item National Science Foundation Graduate Research Fellowship. (2006)\\ 
% Amount: \$138000\\
% Status: \emph{Awarded}
\end{itemize}

\clearpage

\section{\sc Metrics}
h-index = 25; i10-index = 32; Citations = 2562 (as of June 30, 2024)

% \section{\sc Preprints}
% {\footnotesize{${}^\ast$Contributed equally, ${}^\dag$Served as mentor}}
% \begin{enumerate}
%   \reverselabel{35}



% \end{enumerate}



\section{\sc In Revision}
{\footnotesize{${}^\ast$Contributed equally, ${}^\dag$Senior author}}
\begin{enumerate}
  \reverselabel{44}
\item \textbf{Yeakel J.D.}, Hutchinson M.C., Kempes C.P., Koch P.L., Gill J.L., Pires M.M. \emph{Bioenergetic trophic trade-offs determine mass-dependent extinction thresholds across the Cenozoic}. \href{http://arxiv.org/abs/2410.18849}{Preprint}.
\item Dominy N.J., Rosien J., Fannin L., \textbf{Yeakel J.D.}, Malaivijitnond S., Tan A. \emph{Food-washing monkeys recognize the law of diminishing returns}. \href{https://elifesciences.org/reviewed-preprints/98520#}{eLife Reviewed Preprint}. In revision @ eLife.
\item Fannin L.D., Seyoum C.M., Venkataraman V.V., \textbf{Yeakel J.D.}, Janis C.M., Cerling T.E., Dominy N.J. \emph{Behavioral drive during human evolution}. In revision @ Science.
\end{enumerate}


\section{\sc Publications}
{\footnotesize{${}^\ast$Contributed equally, ${}^\dag$Senior author}}
\begin{enumerate}
  \reverselabel{41}

\item Suswaram M., Bhat U., \textbf{${}^\dag$Yeakel J.D.} 2024. \href{https://iopscience.iop.org/article/10.1088/2632-072X/ad5e2e}{\emph{Rising above the noise: the influence of population dynamics on the evolution of acoustic signaling}}. Journal of Physics: Complexity. In Press.

\item Rallings T., Kempes C.P., \textbf{${}^\dag$Yeakel J.D.} 2024. \href{https://www.journals.uchicago.edu/doi/abs/10.1086/731331}{\emph{On the dynamics of mortality and the ephemeral nature of mammalian megafauna.}} The American Naturalist. In Press.

\item Ritwika V.P.S., Gopinathan A., \textbf{${}^\dag$Yeakel J.D.} 2024. \href{https://besjournals.onlinelibrary.wiley.com/doi/10.1111/1365-2656.14070}{\emph{Beyond the kill: The allometry of predation behaviours among large carnivores.}} The Journal of Animal Ecology. 00, 1-13.

\item Valdovinos F.S., Hale K.R.S., Dritz S., Glaum P.R., Mccann K.S., Simon S.M., Thébault E., Wetzel W.C., Wootton K.L.,  \textbf{Yeakel J.D.} 2023. \href{https://www.sciencedirect.com/science/article/pii/S0169534722002841}{\emph{A bioenergetic framework for aboveground terrestrial food webs.}} Trends in Ecology and Evolution. 28, 301-312.

\item ${}^\ast$Kim S.L., \textbf{${}^\ast$Yeakel J.D.}, Balk M.A., Eberle J.J., Zeichner S., Fieman D., Kriwet J. 2022. \href{https://doi.org/10.1098/rspb.2022.0808}{\emph{Decoding the dynamics of dental distributions: insights from shark demography and dispersal}}. Proceedings of the Royal Society B: Biological Sciences. 289, 20220808.

\item Fannin, L. \textbf{Yeakel J.D.}, Venkataraman V.V., Seyoum C., Geraads D., Fashing P., Nguyen N., Fox-Dobbs K., Dominy N.J.  2021. \href{https://doi.org/10.1016/j.palaeo.2021.110393}{\emph{ Carbon and strontium isotope ratios shed new light on the paleobiology and collapse of Theropithecus, a primate experiment in graminivory}}. Palaeogeography, Palaeoclimatology, Palaeoecology. 572, 110393.

\item \textbf{Yeakel J.D.}, Pires M.M., de Aguiar M.A.M., O'Donnell J.L., Guimar\~aes P.R., Gravel D., Gross T. 2020. \href{https://www.nature.com/articles/s41467-020-17164-x}{\emph{Diverse interactions and ecosystem engineering can stabilize community assembly}}. Nature Communications. 11, 3307.

\item Pires M.M., O'Donnell J.L., Burkle L.A., Diaz-Castelazo C., Hembry D.H., \textbf{Yeakel J.D.}, Newman E.A., Medeiros L.P., de Aguiar M.A.M., Guimar\~aes Jr. P.R. 2020. \href{https://esajournals.onlinelibrary.wiley.com/doi/pdf/10.1002/ecy.3080}{\emph{The indirect paths to cascading effects of extinctions in mutualistic networks.}} Ecology. e03080.

\item Gross T., Allhoff K.T., Blasius B., Brose U., Drossel B., Fahimipour A.K., Guill C., \textbf{Yeakel J.D.}, Zeng F. 2020. \href{https://royalsocietypublishing.org/doi/pdf/10.1098/rstb.2019.0455}{\emph{Modern models of trophic meta-communities.}} Philosophical Transactions of the Royal Society B. 375 (1814), 20190455.

\item \textbf{Yeakel J. D.}, Bhat U., Newsome S.D. 2020. \href{https://www.journals.uchicago.edu/doi/10.1086/709019}{\emph{Caching in or falling back at the Sevilleta: the effects of body size and seasonal uncertainty on desert rodent foraging}}. American Naturalist. 169(2) 1-16.

\item Bhat U., Kempes C.P., \textbf{${}^\dag$Yeakel J. D.} 2020. \href{https://doi.org/10.1073/pnas.1907998117}{\emph{Scaling the risk landscape provides insight into optimal life history strategies and the evolution of grazing}}. Proceedings of the National Academy of Sciences.  117(3) 1580-1586.
  {\footnotesize
	\begin{itemize}
		\item See also related PNAS Commentary by J. M. Fryxell. 2020. \href{https://www.pnas.org/content/117/4/1839}{\emph{Life-history models reconstruct mammalian evolution}}. Proceedings of the National Academy of Sciences. 117(4) 1839-1841.
	\end{itemize}
  }
\item de Aguiar M. A. M., Newman E. A., Pires M. M., \textbf{Yeakel J. D.}, Boettiger C., Burkle L. A., Gravel D., Guimar\~aes P. R. Jr, O’Donnell J. L., Poisot T., Fortin M., Hembry D. H. 2019. \href{https://peerj.com/articles/7566/}{\emph{Revealing biases in the sampling of ecological interaction networks}}. PeerJ. 7, e7566.

\item Baiser B., Gravel D., Cirtwill A., Dunne J. A., Fahimpour A. K., Gilarranz L. J., Grochow J. A., Li D., Martinez N. D., McGrew A., Poisot T., Romnuk T. N., Stouffer D. B., Trotta L. B.,Valdovinos F. S., Williams R. J., Wood S. A., \textbf{Yeakel J. D.} 2019. \href{https://onlinelibrary.wiley.com/doi/abs/10.1111/geb.12925}{\emph{Ecogeographical rules and the macroecology of food webs.}} Global Ecology and Biogeography 28(9), 1204-1218.

\item Gibert J. P. \& \textbf{${}^\dag$Yeakel, J. D.} 2019. \href{https://link.springer.com/article/10.1007/s12080-018-0403-2}{\emph{Laplacian matrices and Turing bifurcations: revisiting Levin 1974 and the consequences of spatial structure and movement for ecological dynamics}}. Theoretical Ecology. 169(2), 1–17.

\item Gibert J. P. \& \textbf{${}^\dag$Yeakel J. D.} 2019. \href{https://www.frontiersin.org/articles/10.3389/fevo.2019.00015/full}{\emph{Eco-evolutionary origins of diverse abundance, biomass, and trophic structures in food webs.}} Frontiers in Ecology and Evolution, 7, 1-15.
  
\item Delmas E., Besson M., Brice M.-H., Burkle L., Dalla Riva G. V., Fortin M.-J., Gravel D., Guimar\~aes Jr. P. R., Hembry D., Newman E., Olesen J. M., Pires M., \textbf{Yeakel J. D.}, Poisot T. 2018. \href{https://onlinelibrary.wiley.com/doi/abs/10.1111/brv.12433}{\emph{Analyzing ecological networks of species interactions}}. Biological Reviews. 94(1), 16-36.

\item \textbf{Yeakel J. D.}, Gibert J. P., Gross T., Westley P. A. H., Moore J. W. 2018. \href{https://royalsocietypublishing.org/doi/10.1098/rstb.2017.0018}{\emph{Eco-evolutionary dynamics, density dependent dispersal, and collective behaviour: implications for salmon metapopulation robustness}}. 2018. Philosophical Transactions of the Royal Society B: Biological Sciences. 373(1746), 20170018.

\item \textbf{${}^\ast$Yeakel J. D.}, ${}^\ast$Kempes C. P., ${}^\ast$Redner S. 2018. \href{https://www.nature.com/articles/s41467-018-02822-y}{\emph{Dynamics of starvation and recovery predict extinction risk and both Damuth's law and Cope's rule}}. Nature Communications. 9, 657.

\item Dominy N. J., \textbf{Yeakel J. D.} 2017. \href{https://academic.oup.com/bioscience/article-pdf/67/2/107/10251235/biw133.pdf}{\emph{Frankenstein and the horrors of competitive exclusion}}. Bioscience. 67, 107-110.

\item Novak M., \textbf{Yeakel J. D.}, Noble A. E., Doak D. F., Emmerson M., Estes J. A., Jacob U., Tinker M.T., Wootton J.T. 2016. \href{https://pubs.er.usgs.gov/publication/70173930}{\emph{Characterizing species interactions: What is the community matrix?}} Annual Review of Ecology, Evolution, and Systematics, 47, 409-432.

\item Dominy N. J., \textbf{Yeakel J. D.}, Bhat U., Ramsden L., Wrangham R. W., Lucas P. W. 2016. \href{https://royalsocietypublishing.org/doi/full/10.1098/rsfs.2016.0001}{\emph{How chimpanzees integrate sensory information to select figs}}. Journal of the Royal Society Interface Focus, 6, 20160001.

\item \textbf{Yeakel J. D.}, Bhat U., Elliott Smith E. A., Newsome S. D. 2016. \href{https://www.frontiersin.org/articles/10.3389/fevo.2016.00001/full}{\emph{Exploring the isotopic niche: isotopic variance, physiological incorporation, and the temporal dynamics of foraging}}. Frontiers in Ecology and Evolution, 4, 2188.

\item Crowley B., Melin A. D., \textbf{Yeakel J. D.}, Dominy N. J. 2015. \href{https://www.frontiersin.org/articles/10.3389/fevo.2015.00127/full}{\emph{Do oxygen isotope values reflect the ecology and physiology of Neotropical mammals?}}. Frontiers in Ecology and Evolution, 3, 1-8.

\item Galetti M., Guevara R., Neves C. L., Rodarte R. R., Bovendorp, R. S. Moreira M., Hopkins III, J. B., \textbf{Yeakel J. D.} 2015. \href{https://www.sciencedirect.com/science/article/abs/pii/S000632071500186X}{\emph{Defaunation affects the populations and diets of rodents in Neotropical rainforests}}. Biological Conservation, 190, 2-7.

\item \textbf{Yeakel J. D.}, Dunne, J. A. 2015. \href{https://www.americanscientist.org/article/modern-lessons-from-ancient-food-webs}{\emph{Modern lessons from ancient food webs}}. American Scientist, 103, 188-195.

\item Moore J. W., Beakes M., Nesbitt H. K., \textbf{Yeakel J. D.}, Patterson D., Thompson L., Phillis C., Braun D., Favaro C., Scott D., Carr-Harris C., Atlas W. 2015. \href{https://esajournals.onlinelibrary.wiley.com/doi/full/10.1890/14-0326.1}{\emph{Emergent stability in a large free-flowing watershed}}. Ecology, 96(2), 340-347.

\item \textbf{${}^\ast$Yeakel J. D.}, ${}^\ast$Pires, M. M., ${}^\ast$Rudolf, L., Dominy, N. J., Koch, P. L., Guimar\~aes, P. R., Jr,
\& Gross, T. 2015. \href{https://www.pnas.org/content/early/2015/01/07/1422646112}{\emph{Recovering ecological pattern and process in Ancient Egypt}}. Proceedings of the National Academy of Sciences. 112(3), E240-E240.

\item \textbf{Yeakel J. D.}, Pires, M. M., Rudolf, L., Dominy, N. J., Koch, P. L., Guimar\~aes, P. R., Jr, \& Gross, T. 2014. \href{https://www.pnas.org/content/111/40/14472}{\emph{Collapse of an ecological network in Ancient Egypt}}. Proceedings of the National Academy of Sciences. 111(40), 14472–14477.

\item Moore, J. W., \textbf{Yeakel J. D.}, Peard, D., Lough, J., \& Beere, M. 2014. \href{https://besjournals.onlinelibrary.wiley.com/doi/10.1111/1365-2656.12212}{\emph{Life-history diversity and its importance to population stability and persistence of a migratory fish: steelhead in two large North American watersheds}}. Journal of Animal Ecology. 83(5), 1035-1046.

\item \textbf{Yeakel J. D.}, Moore, J. W., Guimar\~aes, P. R., Jr, \& de Aguiar, M. A. M. 2014. \href{https://onlinelibrary.wiley.com/doi/abs/10.1111/ele.12228}{\emph{Synchronisation and stability in river metapopulation networks}}. Ecology Letters. 17(3), 273–283.

\item \textbf{Yeakel J. D.}, \& Mangel, M. 2014. \href{https://link.springer.com/article/10.1007/s12080-014-0230-z}{\emph{A generalized perturbation approach for exploring stock recruitment relationships}}. Theoretical Ecology. 8(1), 1–13.

\item \textbf{Yeakel J. D.}, Dominy, N. J., Koch, P. L., \& Mangel, M. 2014. \href{https://onlinelibrary.wiley.com/doi/full/10.1111/evo.12240}{\emph{Functional morphology, stable isotopes, and human evolution: a model of consilience}}. Evolution 68, 190–203.

\item \textbf{Yeakel J. D.}, Guimar\~aes, P. R., Jr, Bocherens, H., \& Koch, P. L. 2013. \href{https://royalsocietypublishing.org/doi/10.1098/rspb.2013.0239}{\emph{The impact of climate change on the structure of Pleistocene food webs across the mammoth steppe}}. Proceedings of the Royal Society of London Series B-Biological Sciences 280(1762), 20130239–20130239.

\item \textbf{Yeakel J. D.}, Guimar\~aes, P. R., Jr, Novak, M., Fox-Dobbs, K., \& Koch, P. L. 2012. \href{https://royalsocietypublishing.org/doi/10.1098/rsif.2012.0481}{\emph{Probabilistic patterns of interaction: the effects of link-strength variability on food web structure}}. Journal of the Royal Society Interface 9(77), 3219–3228.

\item Moritz, G. L., Fourie, N., \textbf{Yeakel J. D.}, Phillips-Conroy, J. E., Jolly, C. J., Koch, P. L., \& Dominy, N. J. 2012. \href{https://pubmed.ncbi.nlm.nih.gov/22902370/}{\emph{Baboons, water, and the ecology of oxygen stable isotopes in an arid hybrid zone}}. Physiological and Biochemical Zoology 85(5), 421–430.

\item ${}^\ast$Newsome, S. D., \textbf{${}^\ast$Yeakel J. D.}, Wheatley, P. V., \& Tinker, M. T. 2012. \href{https://academic.oup.com/jmammal/article/93/2/329/919625}{\emph{Tools for quantifying isotopic niche space and dietary variation at the individual and population level}}. Journal of Mammalogy 93(2), 329–341.

\item \textbf{Yeakel J. D.}, Novak, M., Guimar\~aes, P. R., Jr, Dominy, N. J., Koch, P. L., Ward, E. J., et al. 2011. \href{https://journals.plos.org/plosone/article?id=10.1371/journal.pone.0022015}{\emph{Merging resource availability with isotope mixing models: the role of neutral interaction assumptions}}. PLoS ONE 6(7), e22015.

\item \textbf{Yeakel J. D.}, Stiefs, D., Novak, M., \& Gross, T. 2011. \href{https://link.springer.com/article/10.1007/s12080-011-0112-6}{\emph{Generalized modeling of ecological population dynamics}}. Theoretical Ecology 4(2), 179–194.

\item \textbf{Yeakel J. D.}, Patterson, B. D., Fox-Dobbs, K., Okumura, M., Cerling, T., Moore, J., et al. 2009. \href{https://www.pnas.org/content/106/45/19040}{\emph{Cooperation and individuality among man-eating lions}}. Proceedings of the National Academy of Sciences of the USA 106, 19040–19043.

\item Dominy, N. J., Vogel, E. R., \textbf{Yeakel J. D.}, Constantino, P. J., \& Lucas, P. W. 2008. \href{https://link.springer.com/article/10.1007/s11692-008-9026-7}{\emph{Mechanical properties of plant underground storage organs and implications for dietary models of early hominins}}. Evolutionary Biology 35(3), 159–175.

\item \textbf{Yeakel J. D.}, Bennett, N. C., Koch, P. L., \& Dominy, N. J. 2007. \href{https://royalsocietypublishing.org/doi/10.1098/rspb.2007.0330}{\emph{The isotopic ecology of African mole rats informs hypotheses on the evolution of human diet}}. Proceedings of the Royal Society of London Series B-Biological Sciences 274(1619), 1723–1730.

\end{enumerate}

% \section{\sc Papers in preparation}
%
% Ventura, V., C.J. Paciorek, and J.S. Risbey.  Controlling the proportion of falsely-rejected hypotheses when conducting multiple tests with geophysical data.
%
% Ickes, K., C.J. Paciorek, and S. Thomas.  Effects of wild pigs on
% forest demographic processes in Malaysia.


\section{\sc Workshops -- \\ Organizer}
\begin{itemize}
\item Coupled grassland and mammalian community dynamics over ecological and evolutionary timescales II. Justin Yeakel \& Nathaniel Dominy (Organizers). Dartmouth College, May 2016.

\item Complex Life Investment Strategies. Justin Yeakel \& Eric Libby (Organizers). Santa Fe Institute, October 2015.

\item Coupled grassland and mammalian community dynamics over ecological and evolutionary timescales I. Justin Yeakel \& Nathaniel Dominy (Organizers). Santa Fe Institute, September 2015.
\end{itemize}


\section{\sc Workshops \& Internships -- \\ Participant}
\begin{itemize}
\item Archaeopathogens in a Thawing World. Stefani Crabtree (Organizer). Santa Fe Institute. March 2024.

\item Terrestrial Food Web Model Working Group. Fernanda Valdovinos (Organizer). University of California Davis. February 2022.
  
\item International Center for Theoretical Physics (ICTP), Trieste, Italy. Invited to lecture during the 2-week program: \emph{Quantitative Approaches in Ecosystem Ecology}. December 2020. (online-only due to COVID-19 restrictions)

\item NSF-funded Workshop to Advance Ecological Theory. Katriona Shea, Alan Hastings, \& Saran Twombly (Organizers). Pennsylvania State University, October 2019.

\item NIMBioS: Spatiotemporal variation and dynamics in ecological networks I,II,III,IV. Knoxville, TN, June 2015, December 2015, November 2016, February 2019.  

\item Next-generation ecological network theory and application. Phillip P.A. Staniczenko, Fernanda S. Valdovinos, \& Jennifer A. Dunne (Organizers). Santa Fe Institute, November 2018.

\item Predicting the response of host-associated microbiomes to disturbance. Jessica Green \& Ashkaan Fahimipour (Organizers). Santa Fe Institute, August 2016.

\item Gradient-Based Ecological Network Research II. Jennifer Dunne (Organizer). Santa Fe Institute, March 2015.

\item Dynamics On and Of Networks. Jennifer Dunne \& Cris Moore (Organizers). Santa Fe Institute, December 2014.

\item Networks on Networks workshop. Thilo Gross, Barbara Drossel and Ulrich Br\"ose (Organizers). Max Planck Institute for the Physics of Complex Systems (MPIPKS), Dresden Germany, September 2014.

\item Les Ecologists Seminar Series, Simon Fraser University (Organizer). 2013-2014.

\item ESPCA Sao Paulo School on Ecological Networks, Sao Paulo, Brazil, September 16-23 2011

\item Max Planck Institute for the Physics of Complex Systems (MPIPKS), Dresden Germany\\
	Host: Dr. Thilo Gross and the Dynamics of Biological Networks lab, August 2010.

\end{itemize}


\section{\sc Invited Seminars}
\begin{itemize}

\item Bowdoin College. November 2023.

\item University of California Merced - Mathematical Biology Series. September 2022.

\item University of California Davis - Terrestrial Ecosystems Working Group. April 2022.

\item University of S\~au Paulo. October, 2021.
% \\\emph{The dynamics of starvation \& recovery provide insight into Cope's Rule and the constraints of predation}.

\item University of Oldenburg (Helmholtz Inst. for Functional Marine Biodiversity). Sept., 2021.
% \\\emph{The dynamics of starvation \& recovery provide insight into Cope's Rule and the constraints of predation}.

\item University of Maine (School of Biology and Ecology). October, 2019.
% \\\emph{The dynamics of starvation \& recovery provide insight into Cope's Rule and the evolution of grazing}.

\item University of California, Santa Cruz (EEB). April, 2019.
% \\
% \emph{The dynamics of starvation \& recovery provide insight into Cope's Rule and the evolution of grazing}. 

\item Fresno State University. March, 2019.
% \\\emph{Collapse of an ecological network in Ancient Egypt}. 

\item University of California, Riverside. January, 2019.
% \\\emph{The dynamics of starvation \& recovery: Insights into extinction risk, Cope's rule, and life history trade-offs}.

\item UC Merced Mathematical Biology Seminar. 2018.
% \\
% \emph{Perpendicular perspectives:  Exploring the effects of energetic constraints on population and community dynamics}\\
% J.D. Yeakel

\item University of California, Berkeley. December, 2018.
% \\
% \emph{The dynamics of starvation \& recovery:
% Insights into extinction risk, Cope's rule, and life history trade-offs}.

\item University of California, Santa Cruz (Applied Math). November, 2018.
% \\
% \emph{The extinction and assembly of ecological networks}.

\item University of Portland. October, 2018.
% \\
% \emph{It's alive! Competition, extinction, and the ecology of reanimation}.

\item Intelligent Adaptive Systems, University of California Merced. December, 2017.
% \\
% \emph{Extinction, assembly, and engineering in ecological networks}.

\item University of Nebraska, Lincoln. March, 2017.
% \\\emph{Extinction, Cope's rule, and the dynamics of starvation and recovery}. 

\item University of Alaska, Fairbanks. February, 2017.
% \\\emph{Extinction, assembly, and engineering in ecological networks}.

\item Santa Fe Institute Complex Systems Summer School. July, 2016-2017.
% \\\emph{Ecological networks.} 

\item University of California Merced EnviroLunch. March, 2016.
% \\\emph{Modern Lessons from Ancient Food Webs.}

\item University of New Mexico. September, 2015.
% \\\emph{Exploring the isotopic niche.}

\item Santa Fe Institute Complex Systems Summer School. July, 2015.
% \\\emph{Modern lessons from ancient food webs.} 

\item Santa Fe Institute, Santa Fe, New Mexico. December 2014.
% \\\emph{State-dependent interactions in food webs}

\item University of G\"ottingen, G\"ottingen, Germany. September 2014.
% \\\emph{Collapse of an Ancient Egyptian food web}

\item University of California, Merced. March 2014.
% \\\emph{Ecological networks over time and space: from species interactions to community dynamics.} 

\item University of New Mexico. February 2014.
% \\\emph{Ecological networks over time and space: from species interactions to community dynamics.}

\item Santa Fe Institute. January 2014.
% \\\emph{The emergence and evolution of food webs over space and time.} 

\item Oregon State University. January 2014.
% \\\emph{Collapse of an ecological network: reconstructing the decline of an Ancient Egyptian food web.} 

\item University of Wyoming. December 2013.
% \\\emph{Synchronization, stability, and flow in structured metapopulations.} 

\item University of Chicago. December 2013.
% \\\emph{Collapse of an Ancient Egyptian food web.}

\item Washington State University, Pullman. June, 2013.
% \\\emph{Ecological networks over time and space: from species interaction to community dynamics.} 

\item University of California, Santa Cruz. November, 2012.
% \\\emph{Estimating the degree of compensation from short-term fluctuations in fish biomass.} 

\item University of Wyoming. September, 2012.
% \\\emph{Unraveling an ecological network: Reconstructing the decline of ancient Egyptian food webs.} 

\item Simon Fraser University. November, 2011.
% \\\emph{The structure of Mammoth-Steppe food-webs: ecological coherence and the dietary habits of Neanderthals.} 
\end{itemize}

\section{\sc Professional Presentations \& Posters}
\begin{itemize}

% \item American Naturalists Society Meeting. January 2020.
\item Estuarine Connectivity Symposium, University of California Davis. February 2020.\\
\emph{Rescue, extinction, and species interactions on complex `land'scapes}\\
J.D. Yeakel

\item Gordon Research Conference on Plant-Herbivore Interactions. 2019.\\
\emph{Predicting the diets of herbivorous mammals from plant-herbivore trait interactions}\\
T. Rallings, U. Bhat, J. Blois, J.D. Yeakel

\item Gordon Research Conference on Speciation. 2019.\\
\emph{Modeling trait evolution and its population dynamics using dynamical systems}\\
M. Suswaram, J.D. Yeakel, D. Edwards

\item American Physical Society. 2018.\\
\emph{The Fitness Trade-offs of Predation: When to Scavenge and When to Steal}\\
Ritwika VPS, A. Gopinathan, J.D. Yeakel

\item Gordon Research Conference on Unifying Ecology Across Scales. 2018.\\
\emph{Resource investment strategies in uncertain and patchy environments}\\
U. Bhat, C.P. Kempes, J.D. Yeakel

\item Sevilleta LTER Symposium. September 2018.\\
\emph{Linking the persistence of the Sevilleta rodent community to alternative caching and foraging strategies}\\
J.D. Yeakel, S.D. Newsome, U. Bhat

\item Ecological Society of America Annual Meeting. August 2018.\\
\emph{Quantization of ecological interaction networks yields insights into the fundamental processes underlying community assembly}\\
J.D. Yeakel

\item Ecological Society of America Annual Meeting. August 2018.\\
\emph{Fitness Trade-offs of Predation: When to Scavenge and When to Steal}\\
R. VPS, A Gopinathan, J.D. Yeakel

\item Annual Meeting of the Society for the Study of Evolution. 2017.\\
\emph{Relative importance of natural and sexual selection in speciation}\\
M. Suswaram, J.D. Yeakel, D. Edwards

\item Society of Vertebrate Paleontology. 2017.\\
\emph{Mapping mammalian morphological traits to diets with machine learning}\\
T. Rallings, H. Duran, J.D. Yeakel

\item Geological Society of America Annual Meeting. Stable Isotope Workshop. 2017.\\
\emph{Exploring the isotopic niche: challenges to reconstructing modern and paleo food
webs}\\
J.D. Yeakel

\item Geological Society of America Annual Meeting. October 2017.\\
\emph{Extinction, Cope's Rule, and the dynamics of starvation and recovery}\\
J.D. Yeakel

\item James S. McDonnell/Santa Fe Institute Postdocs in Complexity Workshop. July 2017.\\
\emph{Quantization of ecological interactions yields insights into community assembly, dynamics, and engineering}\\
J.D. Yeakel

\item QSB Lightening Talk, UC Merced. September 2017.\\
\emph{Quantization of ecological interactions yields insights into community assembly, dynamics, and engineering}\\
J.D. Yeakel

\item QSB Symposium; University of California, Merced. October 2016.\\
\emph{The dynamics of starvation and recovery}\\
J.D. Yeakel

\item IDEAS Symposium; Simon Fraser University. January 2014.

\item Ecological Society of America Annual Meeting. August 2013.

\item IDEAS Symposium; Simon Fraser University. December 2012.

\item Ecological Society of America Annual Meeting. August 2012.

\item Ecological Society of America Annual Meeting. August 2011.

\item American Fisheries Society. September 2011.

\item 2010 Species Interaction Workshop; Santa Cruz, CA. December 2010.

\item UCSC Graduate Research Symposium. May 2006, 2007, 2008, 2009, 2010, 2011.

\item 2009 Species Interaction Workshop; Stanford CA. December 2009.

\item Carnivore Conference; Defenders of Wildlife. November 2009.

\item Ecological Society of America Annual Meeting. August 2009.

\item 26th Annual Physiological Ecology Meeting. June 2008.

\item American Association of Physical Anthropologists. April 2008.

\item Society of Integrative and Comparative Biology. Jan. 2008.

\item Society of Vertebrate Paleontology. Oct. 2007.

\item American Association of Physical Anthropologists. March 2007.

\item UCSC Plant Sciences Symposium. Feb. 2007, 2009.

\item Applications of Stable Isotope Techniques to Ecological Studies. August 2006.

\item Society of Vertebrate Paleontology. Oct. 2005.

\end{itemize}

\section{\sc Teaching -- Mentorship: \\ Postdocs}
{\footnotesize{${}^\dag$Graduated or Finished}}
\begin{itemize}
\item ${}^\dag$Postdoctoral Fellow Uttam Bhat \hfill 2017-2020
	\begin{itemize}  
		\item[] Currently Research Scientist at Climate LLC, San Francisco CA
	\end{itemize}
\item ${}^\dag$James S. McDonnell Complex Systems Fellow Jean Philippe Gibert \hfill 2016-2019
	\begin{itemize}  
		\item[] Currently Assist. Prof., Duke University, NC
	\end{itemize}  
\item ${}^\dag$Postdoctoral Fellow Jack Hopkins III \hfill 2016
	\begin{itemize}  
		\item[] Currently Assist. Prof., Unity College, ME
	\end{itemize}  
\end{itemize}

\section{\sc Teaching -- Mentorship: \\ Ph.D. Students}
{\footnotesize{${}^\dag$Graduated or Finished}}
\begin{itemize}
  \item Ph.D. Student Anna Carolina de Almeida (Quant. Sys. Biol.) \hfill 2023-present
  \item Ph.D. Candidate Irina Birskis Barros (Quant. Sys. Biol.) \hfill 2018-present
  \item ${}^\dag$Dr. Megha Suswaram (Quant. Sys. Biol.) \hfill 2019-2022
  \item ${}^\dag$Dr. Taran Rallings (Quant. Sys. Biol.) \hfill 2016-2022
  \item ${}^\dag$Dr. Ritwika VPS (Ph.D. in Physics; co-advised w/ A. Gopinathan) \hfill 2016-2021


\end{itemize}


\section{\sc Teaching -- Mentorship: \\ Committee Member}
{\footnotesize{${}^\dag$Graduated or Finished}}
\begin{itemize}
  \item Leo Niehorster-Cook; Ph.D. in Cognitive Sci., UC Merced \hfill 2024-present
  \item Ryan Torres; Ph.D. in Environmental Sys., UC Merced (chair) \hfill 2022-present
  \item Chanuwas Aswamenakul; Ph.D. in Cognitive Sci., UC Merced \hfill 2021-present 
  \item Corey Moser; Ph.D. in Cognitive Sci., UC Merced \hfill 2021-present
  \item Alejandro Perez Velilla; Ph.D. in Cognitive Sci., UC Merced \hfill 2021-present
  \item Luke Fannin; Ph.D. in Ecol., Evol., Environ., \& Soc., Dartmouth College \hfill 2020-present
  \item Ronald Hall; Ph.D. in Quant. Sys. Biol., UC Merced \hfill 2020-present
  \item Shkula Babi; Ph.D. in Quant. Sys. Biol., UC Merced \hfill 2021-2023
  \item ${}^\dag$Brandon Genko; Ph.D. in Environmental Systems, UC Merced \hfill 2021-2023
  \item ${}^\dag$Tanya Strydom; Ph.D. in Quebec Centre for Biodiversity Sci., U. de Montr\'eal \hfill 2021
  \item ${}^\dag$Molly Karnes; Ph.D. in Environ. Sys., UC Merced \hfill 2019-present
  \item ${}^\dag$Amin Boroomand; Ph.D. in Cog. Sci., UC Merced \hfill 2019-2022
  \item ${}^\dag$Jonathan Anzules; Ph.D. in Quant. Sys. Biol., UC Merced (chair) \hfill 2018-2022
  \item ${}^\dag$Gina Palefsky; Ph.D. in Anthropology, UC Merced \hfill 2017-2019
  \item ${}^\dag$Dr. Jesse Wilson; Ph.D. in Environ. Sys., UC Merced \hfill 2017
  \item ${}^\dag$Natalie Graham; Ph.D. in Environ. Sci., Policy, \& Manag., UC Berkeley \hfill 2016-2017
  \item ${}^\dag$Dr. Nathaniel Fox; Ph.D. in Environ. Sys., UC Merced \hfill 2016-2021
  \item ${}^\dag$Dr. Jon Nye; Ph.D. in Environ. Sys., UC Merced \hfill 2016
\end{itemize}



\section{\sc Teaching -- Courses}
{\bf Professor \@ UC Merced (2016-present)} %and Lecturer \@ UC Santa Cruz (2010-2012)
\begin{itemize}
	\item Introduction to Biology (Lower Division UG; Service) \hfill S19, S20
	\item Natural History of Dinosaurs (General Education UG) \hfill S16, S18, S21, F23, S24
	\item Fundamentals of Ecology (Upper Division UG) \hfill S17, F18, F20, Su21, S22, Su22-24
	\item Undergraduate Seminar (ESS 190) \hfill F22
	\item Advanced Topics in Ecology \& Evolutionary Biology (Graduate) \hfill F21
	\item Ecological Dynamics (Graduate) \hfill F17, F19, S23
\end{itemize}

% {\bf Teaching Assistant \@ UC Santa Cruz (2005-2012)}\\
% Conservation Biology, Dept. Environmental Sciences (2011);
% Behavioral Ecology, Dept. Ecology and Evol. Biology (2010);
% Ecology, Dept. Ecology and Evol. Biology (2010, 2011);
% Introduction to Biology, Dept. Ecology and Evol. Biology (2009);
% Human Functional Anatomy, Dept. Anthropology (2007);
% Human Ecology, Dept. Anthropology (2006);
% Natural History of Dinosaurs, Dept. Earth and Planetary Sciences (2006);
% Ecology and Evolution, Dept. Ecology and Evol. Biology (2005);
% Animal and Plant Physiology, Dept. Ecology and Evol. Biology (2005)


\section{\sc Service -- University}
\begin{itemize}
  \item Biology Program Chair \hfill 2023-
  \item Life and Environmental Science ad-hoc graduate group committee \hfill 2023-
  \item Life and Environmental Science Vice-Chair \hfill 2022-23
  \item Faculty Welfare and Academic Freedom Committee member \hfill 2022-23
  \item UC Merced Representative for UC Academic Freedom (UCAF) \hfill 2022-23
  \item UC Merced Representative for the Assembly of the Academic Senate \hfill 2021-22
  \item Climate Crisis Working Group member \hfill 2021-22
  \item Faculty Advisory Committee on Sustainability ex officio member \hfill 2021-22
  \item Action Plan for Advancing Faculty Success Task Force member \hfill 2021
  \item Divisional Council at-large member \hfill 2020-22
  \item ad-hoc Introductory Biology Restructuring Committee \hfill 2021-22
	\item Life Sciences Curriculum Committee \hfill 2019-
	\item Stable isotope lab manager search committee \hfill 2019
	\item Introduction to Biology service course (2x) \hfill 2019-2020
	\item Presidential Postdoctoral Fellow search committee chair \hfill 2019
  \item Host of QSB Seminar Speakers: \emph{A. Hastings}, \emph{A. Fahimipour}, \emph{G. de Leo} \hfill 2016-
	\item Natural Sciences Executive Committee \hfill 2018-2019
	\item QSB Admissions Committee \hfill 2018
	\item QSB Ad-Hoc EEB concentration committee chair \hfill 2018
	\item Committee on Research Computing member \hfill 2017-2020
	\item Developed Ecological Dynamics Graduate course \hfill 2017
	\item Developed Natural History of Dinosaurs GE course \hfill 2016

\end{itemize}

\section{\sc Service -- Editorial}
\begin{itemize}
  \item Review Editor, \emph{eLife} \hfill 2022-
\end{itemize}


\section{\sc Service -- Refereed Journals}
Science,
Nature Communications,
Ecology Letters,
Proceedings of the National Academy of Sciences,
Proceedings of the Royal Society B: Biological Sciences,
Biogeochemistry,
Journal of the Royal Society Interface,
Functional Ecology,
Ecological Applications,
Ecosphere,
PeerJ,
Paleobiology,
Science Advances,
Environmental Modelling \& Software,
Nature Scientific Reports,
Journal of Human Evolution,
Plos Computational Biology,
Fisheries Research,
Theoretical Ecology,
Ecology and Evolution,
Biological Conservation,
IMA Journal of Applied Mathematics,
Current Anthropology,
Methods in Ecology and Evolution,
Quaternary Science Reviews,
Ecography,
Oecologia,
Oikos,
Canadian Journal of Zoology,
Geochimica et Cosmochimica Acta,
PloS One,
Journal of Archaeological Science

\section{\sc Service -- Refereed Grant Agencies}
French National Research Agency (2022),
National Science Foundation-DEB (2016,2018,2019),
Irish Research Council (2018),
European Research Council (2014)

\vspace{0.6cm}

\section{\sc Press}
{\bf Print}\\
Santa Fe Institute (2020) ``More ecosystem engineers create stability, preventing extinctions''; 
Santa Fe Institute (2019) ``Like a video game with health points, energy budgets explain evolutionary body size'';
BBC (2018) ``Size matters when it comes to extinction risk'';
Vice News (2018) ``New Model Predicts the `Ideal' Mammal Is 2.5 Times Bigger Than an Elephant'';
California Academy of Sciences Science News (2016) ``Frankenstein and Extinction'';
Christian Science Monitor (2016) ``A female Frankenstein would lead to humanity's extinction, say scientists'';
National Geographic (July, 2015);
Science NOW (2014) ``Clues to animal extinctions found on the walls of Egyptian tombs'';
Smithsonian (2014) ``Egypt’s mammal extinctions tracked through 6000 years of art'';
NBC News (2014) ``Ancient Egyptian art opens window on mammal extinctions'';
Popular Archeology (2014) ``Study shows how ecology transformed through 6000 years of Egyptian history'';
Nature News (2013) ``Ancient art fills in Egypt’s ecological history'';
Nature News (2009) ``Lions’ taste for human flesh dissected'';
Science News: (2009) ``A body count for two man-eating lions'';
(2014) ``Clues to animal extinctions found on the walls of Egyptian tombs'';
(2009) ``A body count for two man-eating lions'';
Discovery News (2009);
Chicago Sun Times (2009);
Chicago Tribune (2009);
Science Daily (2009);
Telegraph (UK; 2009);
San Francisco Chronicle (2007) ``UC student roots out clues to pre-human species’ diet'';
Nature News (2007) ``Human ancestors went underground for dinner'';
Archaeology Magazine (2007);
Christian Science Monitor (2007)

{\bf Radio}\\
Santa Fe Radio Cafe (2015);
National Public Radio: All Things Considered (2009);
CBC Radio: As It Happens (2009);
Santa Cruz KZSC (2009)

\section{\sc Public Outreach}
\begin{itemize}
  \item Keynote speaker for the Beckman Humor Project at the University of Portland, discussing scientific concepts with humor (October, 2018). \emph{It's alive! Competition, extinction, and the ecology of reanimation} (see assoc. pub. \#23).
  \item Cover feature for \emph{American Scientist} reviewing food web paleoecology (see pub. \#17)
  \item Co-founder of the podcast \emph{Science... Sort of}. \emph{Science... Sort of} is a podcast that discusses ``things that are science, things that are sort of science, and things that wish they were science''. The podcast is designed to introduce and discuss science-based topics in a way that is accessible to both scientists and non-scientists, and has a weekly audience of ca. 2000-5000 listeners.
\end{itemize}

\section{\sc Outdoor\\ Education}
% {\bf Teaching Education/Training}\\
\begin{itemize}
\item LongAcre Expeditions (Trip Leader) \hfill 2003-2004
\item Kent State University Adventure Center (Trip Leader) \hfill 2001-2004
\item National Outdoor Leadership School; Palmer, Alaska (Graduate) \hfill 2001
\end{itemize}



\end{resume}
\end{document}
